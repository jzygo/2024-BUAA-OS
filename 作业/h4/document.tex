\documentclass{article}
\usepackage{listings}
\usepackage{xcolor}
\usepackage[UTF8]{ctex}
\begin{document}
% 用来设置附录中代码的样式

\lstset{
	% basicstyle=\footnotesize,                 % 设置整体的字体大小
	showstringspaces=false,                     % 不显示字符串中的空格
	frame=single,                               % 设置代码块边框
	numbers=left,                               % 在左侧显示行号
	% numberstyle=\footnotesize\color{gray},    % 设置行号格式
	numberstyle=\color{darkgray},               % 设置行号格式
	backgroundcolor=\color{white},              % 设置背景颜色
	keywordstyle=\color{blue},                  % 设置关键字颜色
	commentstyle=\it\color[RGB]{0,100,0},       % 设置代码注释的格式
	stringstyle=\sl\color{red},                 % 设置字符串格式
	columns=flexible
}

\lstdefinestyle{C}{
	language        =   C, % 语言选Python
	basicstyle      =   \zihao{-5}\ttfamily,
	numberstyle     =   \zihao{-5}\ttfamily,
	keywordstyle    =   \color{blue},
	keywordstyle    =   [2] \color{teal},
	stringstyle     =   \color{magenta},
	commentstyle    =   \color{red}\ttfamily,
	breaklines      =   true,   % 自动换行,建议不要写太长的行
	columns         =   fixed,  % 如果不加这一句,字间距就不固定,很丑,必须加
	basewidth       =   0.5em,
}
{\LARGE $Q1$}
\lstset{language=C}
\begin{lstlisting}
int count = 0;
semaphore mutex = 1;
semaphore rw = 1;
semaphore w = 1;

writer() {
	while(1) {
		P(w);
		P(rw);
		writing;
		V(rw);
		V(w);
	}
}

reader() {
	while(1) {
		P(w);
		P(mutex);
		if(count==0) {
			P(rw);
		}
		count++;
		V(mutex);
		V(w);
		reading;
		P(mutex);
		count--;
		if(count==0) {
			V(rw);
		}
		V(mutex);
	}
}
\end{lstlisting}

{\LARGE $Q2$}
\lstset{language=C}
\begin{lstlisting}
int count = 0;
semaphore number = 6;
semaphore waiting = 0;
semaphore mutex = 1;

customer() {
	P(number);
	P(mutex);
	if(count==5) {
		p(waiting);
	}
	count++;
	V(mutex);
	eating;
	P(mutex);
	count--;
	if(count == 0) {
		V(waiting);
	}
	V(mutex);
	V(number);
}
\end{lstlisting}
	
{\LARGE $Q3$}

{\large ($1$):}
要证明 \(((A \to B) \to A) \to A\),我们可以使用题目中给出的定理和一些额外的逻辑定律:肯定后件律、蕴含词分配律、换位律。

1. \( A \to A \)(自反律)
2. \((P \to Q) \to (\neg Q \to \neg P)\)(换位律)
3. \( Q \to (P \to R), P \vdash Q \to R \)(假言三段论)
4. \( P \to Q, Q \to R \vdash P \to R \)(假言推理)
5. \( A \to B, A \vdash B \)(肯定后件律)
6. \( (A \to B) \to ((A \to (B \to C)) \to (A \to C)) \)(蕴含词分配律)

1. **目标**:证明 \(((A \to B) \to A) \to A\)

2. 设 \(P = (A \to B) \to A\) 和 \(Q = A\)

3. 使用假言三段论(定理3):
\[
Q \to (P \to Q), P \vdash Q \to R
\]

4. 具体应用:
1. \(A \to (P \to A)\) (因为 \(Q = A\) 和 \(R = A\))
2. \(P = (A \to B) \to A\)

5. 根据蕴含词分配律(定理6),我们有:
\[
(A \to B) \to ((A \to (B \to A)) \to (A \to A))
\]

6. 根据换位律(定理2),我们有:
\[
(A \to B) \to A \Rightarrow \neg A \to \neg ((A \to B) \to A)
\]

7. 根据肯定后件律(定理5),我们有:
\[
A \to A
\]

8. 最后,将这些结合在一起:
\[
((A \to B) \to A) \to A
\]

以上步骤使用了公理3两次,并结合其他定理完成了证明。这样我们就完成了这个命题逻辑的证明。
信号量(Semaphore)是一个整数变量,可理解为一个计数器,用于控制多个线程或进程对共享资源的访问。根据信号量的取值不同,可以有两种类型:二进制信号量(Binary Semaphore):取值只能为0或1,通常用作互斥锁(Mutex)。计数信号量(Counting Semaphore):取值可以是一个非负整数,表示可用的资源数量。

P操作(Proberen,荷兰语动词"试图"的含义):P操作对应于请求或等待一个资源。当线程或进程想要获取资源时,它会执行P操作。如果信号量的值大于0,表示有资源可用。执行P操作会将信号量的值减去1,然后该线程或进程会持续其执行。如果信号量的值为0,则没有可用资源。执行P操作的线程或进程将被阻塞,直到信号量的值再次变得大于0(即有其他线程或进程释放资源)。

V操作(Verhogen,荷兰语动词"增加"的含义):
V操作对应于释放一个资源。当线程或进程完成对资源的使用后,它会执行V操作。V操作将信号量的值加1,表示一个资源单元变为可用状态。如果有其他线程或进程正在P操作中被阻塞,增加了信号量的值可能导致等待的线程或进程被唤醒,以便能够继续执行并访问资源。

物理意义的解释:可以把信号量想象为一个有限容量的停车场,P操作像是一辆车进入停车场,如果有空余车位(信号量大于0),车辆就进入并占用一个车位(信号量减1)。如果没有空车位(信号量为0),车辆就在入口等待。而V操作就像是一辆车离开停车场,释放了一个车位(信号量加1),如果有车辆在等待,它们就可以占用这个刚释放的车位了。

{\large ($2$):}
\lstset{language=C}
\begin{lstlisting}
int count = 0;
semaphore mutex = 1;
semaphore waiting = 0;
semaphore in = 1;

exployer() {
	P(mutex);
	count++;
	if(count==5) {
		V(mutex);
		for(int i = 1;i <= 4;i++)
			V(waiting);
	}
	else {
		V(mutex);
		P(waiting);
	}
	P(in);
	if(count == 1)
		closeDoor();
	count--;
	V(in);
}
\end{lstlisting}
	
{\LARGE $Q4$}
\lstset{language=C}
\begin{lstlisting}
int count = 0;
semaphore insert = 1;
semaphore delete = 1;
semaphore mutexWrite = 1;

reader() {
	P(mutexWrite);
	if(count==0) {
		P(delete);
	}
	count++;
	V(mutexWrite);
	reading;
	P(mutexWrite);
	count--;
	if(count==0) {
		V(delete);
	}
	V(mutexWrite);
}

inserter() {
	P(insert);
	P(mutexWrite);
	if(count==0) {
		P(delte);
	}
	count++;
	V(mutexWrite);
	inserting;
	P(mutexWrite);
	count--;
	if(count==0) {
		V(delete);
	}
	V(mutexWrite);
	V(insert);
}

deleter() {
	P(delete);
	deleting;
	V(delete);
}
\end{lstlisting}
\end{document}